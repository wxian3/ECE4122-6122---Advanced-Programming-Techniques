\documentclass[10pt]{article}
\usepackage{times,epsfig}
\setlength{\textwidth}{6.5in}
\setlength{\textheight}{9in}
\setlength{\oddsidemargin}{0.0in}
\setlength{\evensidemargin}{0.0in}
\setlength{\topmargin}{-0.5in}
\usepackage{amsmath}
\input{../teach}
\input{../TAName}
\begin{document}
\header{Project 3 -- Templated Vector Class}
{Assigned: October 4, 2016}
{Due: October 28, 2016, 11:59pm}
{ECE4122/6122}
{Fall Semester, 2016}

\paragraph{Overview}
For this project, we will implement a templated class called {\tt Vector}
that is essentially a variable length array.  The {\tt Vector}
can grow as necessary as new elements are added to the array.  Further
we will implement a iterator called {\tt VectorIterator} that exists to
``refer to'', or ``point to'' an existing element in the {\tt Vector}.
The required API's are specifed in the provided {\tt Vector.h} file.

\paragraph{Copying the Project Skeletons}
\begin{enumerate}
\item Log into {\tt jinx-login.cc} using {\tt ssh} and your prism log-in name.
\item Copy the files from the ECE8893 user account using the following
command:
\begin{verbatim}
/usr/bin/rsync -avu /nethome/ECE6122/Vector .
\end{verbatim}
Be sure to notice the period at the end of the above command.
\item Change your working directory to {\tt Vector}
\begin{verbatim}
cd Vector
\end{verbatim}
\item Copy the provided {\tt Vector-skeleton.cc} to {\tt Vector.cc} as follows:
\begin{verbatim}
cp Vector-skeleton.cc Vector.cc
\end{verbatim}
\item If you are an undergraduate, comment out the definition of
{\tt GRAD\_STUDENT} in {\tt Vector.h}.
\item Then edit {\tt Vector.cc} to implement your {\tt Vector} and
{\tt VectorIterator}
\item Compile your code using {\tt make} as follows:
\begin{verbatim}
make
\end{verbatim}
\item This builds a program called {\tt testVec} that has eight
different test cases to test your implementation.
\item Once you have gotten the program compiled and ready to test,
you can begin running the tests.  Running {\tt testVec} with no arguments
runs all tests; running with a single argument is assumed to be the
test number you want to run.  Grad students must run all eight tests,
undergrads only need tests 1-6.  You should consider debugging by running
one test at a time.
\end{enumerate}

\paragraph{Resources}
\begin{enumerate}
\item {\tt Vector-skeleton.cc} is a starting point for your program.
\item {\tt Vector.h} describes the classes and required API.
\item {\tt testVec.cc} is the test program to test your implementation.  Do not change this program.
\item {\tt String.h} and {\tt String.cc} are used by the testing program.
You should not need to edit either of these.
\item {\tt Makefile} is the makefile for the testVec program.
\item {\tt expected-alltests.txt} is the expected output from running
all tests.
\end{enumerate}

\paragraph{Checking for Memory Leaks}
An important requirement for this assignment is to properly manage memory
and to have NO MEMORY LEAKS.  On the jinx--login system, you can run
the {\tt valgrind} program, which give detailed memory usage statistics
and leak statistics.  Run {\tt valgrind} as follows:

\begin{verbatim}
valgrind --tool=memcheck testVec (your test number here)
\end{verbatim}

At the end of the execution, {\tt valgrind} prints a summary of the
memory leaks.  You are hoping for the message
{\tt All heap blocks were freed -- no leaks are possible}.
If you have memory leaks, you will only get half credit for the particular
test (see the section below on grading).

\paragraph{In--Place new and delete}
Recall that we said in class that the {\tt new} operator, by default,
does two separate and independent things. First is to find some memory
to store the new object, and second is to call the constructor for
the object.  In this assignment, you will often want to call a constructor,
but to use existing memory.  This is called an {\em in--place new}.
The syntax is:

{\tt new (address) constructor}

The {\tt address} is the address of an existing memory region where
the new object will be stored.  The {\tt constructor} is the
constructor (default, copy, etc) to use.

Similarly, the {\tt delete} operator does two independent things.
First is to call the destructor for the object, and the second is to
free the memory associated with the object.  Again, in this assignment
you will frequently want to call the destructor for an object, but do not
want the memory freed.  Given an object of type $T$, the syntax for
in--place delete is:

{\tt object.\symbol{126}T();}  or {\tt objPointer->\symbol{126}T();}

Here, {\tt object} is a variable of type $T$ and {\tt objPointer}
is a pointer to an object of type $T$.


\paragraph{Grading}
For undergrads, there are six individual tests, worth 20 points each.
For graduates, there are eight test worth 15 points each. In both 
cases, the total possible is 120 points (this is a very difficult
assignment!).  If you get the right answer but have memory leaks, you
will get half credit for the particular test.

\paragraph{Using the GNU Debugger}
The GNU debugger {\tt gdb} is invaluable in debugging this assignment.
We will discuss {\tt gdb} and other issues for this assignment in the
next class lecture, Wed Oct 19.

\paragraph{Turning in your Project.}
The system administrator for the jinx cluster has created a script
that you are to use to turn in  your project.  The script is called
{\tt riley-turnin} and is found in {\tt /usr/local/bin}, which should
be in the search path for everyone.  From your {\bf home directory}
(not the Vector subdirectory), enter:
\par
{\tt turnin-ece6122 Vector}.

or

{\tt turnin-ece4122 Vector}.

\par
This automatically copies everything in your {\tt Vector}
directory to a place that I can access (and grade) it.


\end{document}
