\documentclass[10pt]{article}
\usepackage{times,epsfig}
\setlength{\textwidth}{6.5in}
\setlength{\textheight}{9in}
\setlength{\oddsidemargin}{0.0in}
\setlength{\evensidemargin}{0.0in}
\setlength{\topmargin}{-0.5in}
\usepackage{amsmath}
\input{../teach}
\input{../TAName}
\begin{document}
\header{Project 1 -- The Distributed Two--Dimensional Discrete Fourier Transform}
{Assigned: Mon August 29, 2016}
{Due: Mon September 12, 2016, 11:59pm}
{ECE6122/4122}
{Fall Semester, 2016}

\def\td{two--dimensional}
\def\od{one--dimensional}
Given a \od\ array of complex or real input values of length $N$,
the Discrete Fourier Transform consists af an array of size $N$ computed
as follows:

\begin{equation}
\label{Eq:DFT}
H[n] = \sum_{k=0}^{N-1} W^{nk}h[k]
\hspace{0.5in}\mbox{where}\ \ 
W = e^{-j2\pi/N} = \mbox{cos}(2\pi/N)-j\mbox{sin}(2\pi/N)
\hspace{0.5in}\mbox{where}\ \ 
j = \sqrt{-1}
\end{equation}
For all equations in this document, we use the following notational
conventions.
$h$ is the discrete--time sampled signal array.
$H$ is the Fourier transform array of $h$. 
$N$ is the length of the sample array, and is always
assumed to be an even power of 2.
$n$ is an index into the $h$ and $H$ arrays, and is always in the
range $0 \ldots (N-1)$.
$k$ is also an index into $h$ and $H$, and is the summation
variable when needed.
$j$ is the square root of negative one.

The above equation clearly requires $N^2$ computations, and as $N$
gets large the computation time can become excessive.  There are
a number of well--known approaches that reduce the comutation time
considerably, but for the purpose of this assignment you can just
use the simple double summation shown above.

Given a \td\ matrix of complex input values, the \td\
Fourier Transform can be computed with two simple steps.  First, the
\od\ transform is computed for each row in the matrix 
individually.  Then a {\em second} \od\ transform is done on each
{\em column} of the matrix individually.
Note that the transformed values from the
first step are the inputs to the second step.

If we have several CPU's to use to compute the 2D DFT, it is easy to see
how some of these steps can be done simulataneously.  For example, if we
are computing a 2D DFT of a 16 by 16 matrix, and if we had 16 CPUs available,
we could assign each of the
16 CPU's to compute the DFT of a given row.  In this simple example,
CPU 0 would compute the \od\ DFT for row 0, CPU 1 would compute for row 1,
and so on.  If we did this, the first
step (computing DFT's of the rows) should run 16 times faster than when
we only used one CPU.

However, when computing the second step, we run into difficulties.  
When CPU 0 completes the \od\ DFT for row 0, it would presumably be ready
compute the 1D DFT for {\em column} 0.  Unfortunately, the computed results
for all other columns are not available to CPU 0 easily.  We can solve this
problem by using {\em message passing}.  After each CPU completes the 
first step (computing 1D DFT's for each row), it must send the required
values to the other processes using {\em MPI}.  In this example, CPU 0
would send to CPU 1 the computed transform value for row 0, column 1, and
send to CPU 2 the computed transform value for row 0, column 2, and so on.
When each CPU has received $k$ messages with column values ($k$ is the
total number of columns in the input set), it is then ready to compute
the column DFT.

Finally, each CPU must report the final result (again using message passing)
to a designated CPU responsible for collecting and printing the final
transformed value.  Normally, CPU 0 would be chosen for this task, but in 
fact any CPU could be assigned to do this.


We are going to use 16 CPUs in the {\em deepthought} cluster to perform
the 2d DFT using distributed computing.
%% The job scheduling on {\em deepthought}
%% is done by the {\em Torque Resource Manager}. This is described on the
%% {\em deepthought} web page at:

{\tt http://support.cc.gatech.edu/facilities/instructional-labs/deepthought-cluster/ \\
      how-to-run-jobs-on-the-deepthought-cluster}

{\em For now ignore the discussion about using {\tt qsub} to submit job requests.
We will discuss this in class. }
%% A correct job scheduling script is provided for you in file {\tt runDFT.sh}.
%% Instructions for submitting jobs are given below.

\paragraph{Copying the Project Skeletons}
\begin{enumerate}
\item Log into {\tt deepthought19.cc} using {\tt ssh} and your prism log-in name.
\item Copy the files from the ECE6122 user account using the following
command:
\begin{verbatim}
/usr/bin/rsync -avu /nethome/ECE6122/FourierTransform2D .
\end{verbatim}
Be sure to notice the period at the end of the above command.
\item Change your working directory to {\tt FourierTransform2D}
\begin{verbatim}
cd FourierTransform2D
\end{verbatim}
\item Copy the provided {\tt fft2d-skeleton.cc} to {\tt fft2d.cc} as follows:
\begin{verbatim}
cp fft2d-skeleton.cc fft2d.cc
\end{verbatim}
\item Then edit {\tt fft2d.cc} to implement the transform.
\begin{enumerate}
\item Implement a simple \od\ DFT using the double summation approach
in the equations above.
\item Use MPI send and receive to send partial
information between the 16 processes.
\item Use CPU at rank zero to collect the final transformed values
from all other CPU's, and write these results to a file called
{\tt MyAfter2D.txt} using the {\tt SaveImageData} method in
class {\tt InputImage}.
\end{enumerate}
\item Compile your code using {\tt make} as follows:
\begin{verbatim}
make
\end{verbatim}
\end{enumerate}
%% \item Once you have gotten the fft2d program compiled and ready to test,
%% edit the {\tt runDFT.sh} script to change {\tt YourLastNameHere}
%% to give your last name.  This is how your jobs are identified to the
%% job scheduler.
%% \item Test your solution
%% with the provided inputs.  Testing is done by submitting your job to
%% the scheduler using {\tt qsub}.
%% \begin{verbatim}
%% qsub runDFT.sh
%% \end{verbatim}
%% \item You can check the status of your submitted jobs by using {\tt qstat}
%% \begin{verbatim}
%% qstat runDFT.sh
%% \end{verbatim}
%% \item When your submitted job completes, you will have two files created
%% in your workign directory with the outputs from the job and any errors produced by your job (hopefully
%% empty).
%% \end{enumerate}

\paragraph{Resources}
\begin{enumerate}
\item {\tt fft2d-skeleton.cc} is a starting point for your program.
%% \item {\tt runDFT.sh} is the job submission script to use.  Be sure to change
%% the {\tt -N} parameter to your name.
\item {\tt Complex.cc} and {\tt Complex.h} provide a completed C++
object containing a complex (real and imaginary parts) value.
\item {\tt Makefile} is a file used by the {\tt make} command to
build fft2d.
\item {\tt Tower.txt} is the input dataset, a 256 by 256 image of the
Tech tower in black and white.
\item {\tt after1D.txt} is the expected value of the DFT
after the initial \od\ transform on each row, but before the column
transforms have been done. This is for debugging only as the assignment does
not need to write the \od\ transformed results.
\item {\tt after2D.txt} is the expected output dataset after the two dimensional
transformation, 
a 256 by 256 matrix of the transformed values.
\item {\tt InputImage.cc} and {\tt InputImage.h} that will ease the reading
of the input data.  This object has several useful functions to help with
managing the input data.
\item DO NOT ASSUME that the MPI size will always be 16.  Instead implement
the program with a variable (perhaps {\tt nCPUs}) that contains the
MPI size.
\begin{enumerate}
\item The {\tt InputImage} constructor, which has a {\tt char*} argument
specifying the file name of the input image.
\item The {\tt GetWidth()} function that returns the width of the image.
\item The {\tt GetHeight()} function that returns the height of the image.
\item The {\tt GetImageData()} function returns a one-dimensional array
of type {\tt Complex} representing the original time-domain input values.
\item The {\tt SaveImageData} function writes a file containing the
transformed data.
\item The {\tt SaveImageDataReal} function writes a file with only
the real part of the image transformed data.
\end{enumerate}
\end{enumerate}

\paragraph{Output File Naming Convention}
In order to ease the grading procedure, you must write the results of
the 2D transform on a file named {\tt MyAfter2D.txt}.  
For debugging, if you want the results of the 1D transforms, write
the results to a file named {\tt MyAfter1D.txt}.
For graduate students also computing the reverse transform, write those results
to a file named {\tt MyAfterInverse.txt}.

\paragraph {Graduate Students only}.
After the 2D transform has been completed, use MPI again to calculate the
{\em Inverse} transform, and write the results to file {\tt MyAfterInverse.txt}.
Use the {\tt SaveImageDataReal} function to write results. This function writes
the real part only (as the imaginary parts should be zero or near zero after
the inverse.

\paragraph{Turning in your Project.}
%% Your assignment will be turned in automatically on the due date.
%% Be sure to put your code in subdirectory {\tt FourierTransform2D}
%% in your home directory on {\tt deepthought19.cc}
Details about turning in your program will be provided.  

\paragraph{Some thoughts on implementing the 2D DFT}
\begin{enumerate}
\item Consider using Rank 0 as the traffic cop to handle the data
collection and dissemination.  This means you would actually need
17 ranks, using one (rank 0) as the coordinator, and the other 
16 ranks as the ``worker'' processes.
\item After reading in the original {\tt Tower.txt}, create a second
array of size (width * height) which represents the $H$ array (the
transformed data).  We need this because the simple DFT algorithm
given above cannot transform {\em in-place}. In other words, the
$H$ array and the $h$ array are different areas of memory.
\item Clearly a working \od\ DFT is needed before a correct \td\ 
DFT can be implemented.  Consider using a single CPU (no MPI) to
read the {\tt Tower.txt} file (using {\tt InputImage}), and doing
a \od\ DFT on all rows. Then save the resulting file (again
using the {\tt InputImage.SaveImageData}) and comparing to 
{\tt after1d.txt}.
\item Once the \od\ DFT is working, use MPI as described above, and
assign each CPU the correct number of rows, and what row they
should start on.  Assuming {\tt nCpus} is the variable containing the
number of CPU's, {\tt nRows} is the number of rows in the image, and
{\tt myRank} is the rank number of this CPU, then the number of rows
per CPU is {\tt nRows / nCpus}, and the starting row number for
each CPU is {\tt nRows / nCpus * myRank}.
\item After computing the \od\ DFT on each assigned row use MPI
to send information to all other CPU.  Each CPU will need to send
{\tt nCpus - 1} messages, one message to all other CPUs.  Also you will
likely need a separate array of type Complex, as the data to be send
is not necessarily in sequential memory locations.
\item You shoudl consider a mix of blocking and non-blocking MPI
calls to move the data around the various ranks as needed.
\item Also consider using the Tag field in the send and receive
calls to get the received data in the proper location as you move
the rows and columns around.
\item After receiving information from all of your peers (and storing
that information in the $H$ array in the right place), perform the
column-wise \od\ DFT on each of your columns.
You might need yet another array of size (width * height) for this
as well, although you can actually use a smaller array (each CPU only
uses operates on a subset of all the columns).
\item After the second set of transforms, use MPI to send all computed
information to the master (CPU rank 0).  You can use non-blocking
send here, as we are sure CPU 0 will eventually call {\tt recv}.
\item Finally, write out the final 2d transform using {\tt SaveImageData}.
\end{enumerate}
\end{document}
